%导言区
\documentclass[UTF8]{article}
\usepackage{ctex}
\usepackage{amsmath}
\date{\today}

%正文区 
\begin{document}
	\section{简介}
	\section{行内公式}
	\subsection{美元符号}
	$a+b = b+a$
	\subsection{小括号}
	\(a+b=b+a\)
	\subsection{math环境}
		\begin{math}
			a*b=b*a
		\end{math}
	\section{上下标}
	\subsection{上标}
	$x^{20}+2x=0$
	
	$x^{x^{20}+2x=0}+2x=0$
	\subsection{下标}
	$a_0, a_1$
	
	$a_0 , a_1, ... ,a_{100} $
	\section{希腊字母}
	\section{数学函数}
	$\log$
	$\sin$
	$\cos$
	$\arcsin$
	$\ln$
	
	$\sin^2 x+\cos^2 x =1$
	
	$\log^2 x+\ln^2 x =???$
	
	$y=log_2 x$
	
	$\sin^{-1}x$
	
	sqrt 用于排版根式
	
	$\sqrt{2}$
	
	$\sqrt[4]{2}$
	
	\section{分式}
	$\frac{1}{2}$
	
	$\frac{1}{1+\frac{1}{x}}$
	
	\section{行间公式}
	\subsection{美元符号}
	双\$ 符号 的公式会自动化居中
	$$a+b=b+a$$
	
	\subsection{自动编号的equation环境}
	假设公式如\ref{lin}
	\begin{equation}\label{lin}
		a+b+c=0
	\end{equation}
	\subsection{美元符号}
	
	\subsection{不编号的equation*环境}
	此环境需要 添加 amsmath包
	\begin{equation*}
	a+b+c=0
	\end{equation*}
	\subsection{美元符号}
	
\end{document}
	

	