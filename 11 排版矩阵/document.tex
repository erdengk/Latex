%导言区
\documentclass[UTF8]{article}
\usepackage{ctex}
\usepackage{amsmath}
\date{\today}

%正文区 
\begin{document}
	\section{矩阵}
	需要引入 amsmath 包
	\[
	\begin{matrix}
		0 & 1\\
		1 & 9 
	\end{matrix} \qquad 
	\begin{pmatrix}
	0 & 1\\
	1 & 9 
	\end{pmatrix}\qquad
	\begin{bmatrix}
	0 & 1\\
	1 & 9 
	\end{bmatrix}\qquad
	\begin{Bmatrix}
	0 & 1\\
	1 & 9 
	\end{Bmatrix}\qquad
	\begin{vmatrix}
	0 & 1\\
	1 & 9 
	\end{vmatrix}\qquad
	\begin{Vmatrix}
		0 & 1\\
		1 & 9 
	\end{Vmatrix}\qquad
	\]
	
	也可以使用上下标,矩阵内不可以有空格。
	\[
	A = \begin{bmatrix}
		a_{11}^2 & a_{12} ^2& a_{13}^2 \\
		0 & a_{22}^2 & a_{23}^2 \\
		0 & 0 & a_{24}^2 \\
	\end{bmatrix}
	\]
	
	省略号 dots 、 vdots 、 ddots 。n x n 可以用 n 斜杠 + times n实现
	
	\[
	A = \begin{bmatrix}
	a_{11}^2 & \dots & a_{1n}^2 \\
	 & \ddots & \vdots \\
	0 &  & a_{nn} \\
	\end{bmatrix}_{n \times n}
	\]
	
	分块矩阵的实现
	
		\[
	A = \begin{bmatrix}
	\begin{bmatrix}
	a_{11}^2 & \dots & a_{1n}^2 \\
	& \ddots & \vdots \\
	0 &  & a_{nn} \\
	\end{bmatrix}_{n \times n} & \dots & a_{1n}^2 \\
	& \ddots & \vdots \\
	0 &  & a_{nn} \\
	\end{bmatrix}_{n \times n}
	\]
	
	行内矩阵的实现
	\begin{math}
	\left(
		\begin{smallmatrix}
			0 & 1\\
			1 & 9 
		\end{smallmatrix}
	\right)%手动加上左右括号
	\end{math}

	
	还有三角矩阵、数字分块矩阵、数组排版超复杂矩阵详情见视频(4min处)。
	
	
	
	
\end{document}
	

	